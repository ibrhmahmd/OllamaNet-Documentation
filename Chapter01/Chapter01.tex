\def\chapdir{./ChapterIntro}

\chapter{Introduction} \label{ch:intro}

This thesis template should provide you with enough \LaTeX{} code to get writing immediately on your thesis, without having to learn too much \LaTeX{} up front. You may wish to save snippets from the template, or the entire template somewhere separate before you overwrite too much of it if you're unfamiliar with \LaTeX{}, so you can refer back to the source and the PDF it produces if you get stuck. You can cite like this \cite{Douillard2009}

\section{Section in Introduction}
This thesis template should provide you with enough \LaTeX{} code to get writing immediately on your thesis, without having to learn too much \LaTeX{} up front. You may wish to save snippets from the template, or the entire template somewhere separate before you overwrite too much of it if you're unfamiliar with \LaTeX{}, so you can refer back to the source and the PDF it produces if you get stuck.

This thesis template should provide you with enough \LaTeX{} code to get writing immediately on your thesis, without having to learn too much \LaTeX{} up front. You may wish to save snippets from the template, or the entire template somewhere separate before you overwrite too much of it if you're unfamiliar with \LaTeX{}, so you can refer back to the source and the PDF it produces if you get stuck.

This thesis template should provide you with enough \LaTeX{} code to get writing immediately on your thesis, without having to learn too much \LaTeX{} up front. You may wish to save snippets from the template, or the entire template somewhere separate before you overwrite too much of it if you're unfamiliar with \LaTeX{}, so you can refer back to the source and the PDF it produces if you get stuck.


% Note that it's not in a table environment, since this is NOT a floating environment
\begin{center}
    % Use the supertabular commands rather than the usual caption/etc commands, since they are
    % inserted by supertabular
    \tablecaption[Page-Spanning `Super Table']{This table is especially long, so it's been turned
        into a \texttt{supertabular} environment allowing it to span multiple pages.
        \label{tab:supertabex2}}
    % headers and footers can be defined to repeat on each page
    % firsthead and lasttail will be used INSTEAD OF the basic head/tails where appropriate
    \tablefirsthead{%
        \toprule
        \textbf{first} & \textbf{second} & \textbf{RHS}\\
        \midrule[1pt]
    }
    \tablehead{%
        \multicolumn{3}{l}{\small\sl continued from previous page}\\
        \toprule
        \textbf{first} & \textbf{second} & \textbf{RHS}\\
        \midrule
    }
    \tablelasttail{%
        \bottomrule
    }
    \tabletail{%
        \bottomrule
        \multicolumn{3}{l}{\small\sl continued on next page}\\
    }
    % Note that this is JUST the data in the table, since supertabular constructs multiple tables
    % using the firsthead/head/lasttail/tail definitions
    \begin{supertabular}{c @{$\times$} c @{$=$} c}
        1 & 1 & 1 \\
        1 & 2 & 2 \\
        1 & 3 & 3 \\
        1 & 4 & 4 \\
        1 & 5 & 5 \\
        1 & 6 & 6 \\
        1 & 7 & 7 \\
        1 & 8 & 8 \\
    \end{supertabular}
\end{center}

\begin{figure}
    \centering
    \includegraphics[width=0.8\textwidth]{./Chapter01/figures/Database_Approach.jpg}
    \caption{Database Approach}
\end{figure}

\subsection{Subsection in Introduction}





