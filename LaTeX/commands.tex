%!TEX root = ../Thesis.tex

% Some specific formatting commands for types of text

% highlight for changes
\definecolor{highlightcolor}{rgb}{1,0,0}    % remove highlights by setting to {0,0,0}

% Todo notes for draft
% Don't forget that some todo items are markers to actual words in the text that are perhaps  
% inaccurate, and so need to be replaced --- don't just redefine the \todo macro to \empty !
\definecolor{todocolour}{rgb}{1,0.3,0.2}
\definecolor{TODOcolour}{rgb}{1,0,0}
\newcommand{\todo}[2][brackets]{\textsf{\textcolor{todocolour}{%
    \ifthenelse{\equal{#1}{brackets}}{[TODO: #2]}{TODO: #2}%
}}}

% Other things to simplify remembering
\newcommand{\italics}{\textit}
\newcommand{\smallcaps}{\textsc}
\newcommand{\fixedwidth}{\texttt}
\newcommand{\sans}{\textsf}

% Nice marginpars
\let\oldmarginpar\marginpar
\setlength{\marginparwidth}{0.7in}
\renewcommand\marginpar[1]{\-\oldmarginpar[\raggedleft\scriptsize #1]%
{\raggedright\scriptsize #1}}

% Some I may not actually use
\newcommand{\usecase}[1]{\emph{#1}}
\newcommand{\role}[1]{\emph{#1}}
\newcommand{\comp}[1]{\textsc{#1}}
\newcommand{\iface}[1]{\emph{#1}}
\newcommand{\term}[1]{\emph{#1}}
\newcommand{\robot}[1]{\emph{#1}}
\newcommand{\code}[1]{\texttt{#1}}
\newcommand{\bigO}[1]{\ensuremath{\mathcal{O}\bigl(#1\bigr)}}
\newcommand{\bigOpar}[1]{\ensuremath{\mathcal{O}\left(#1\right)}}
\newcommand{\latin}{\emph}                  % (may or may not want latin text italicised)

% Random general stuff...
\newcommand{\emailaddr}[1]{\href{mailto:#1}{#1}}

% autoref case is set here
\def\figureautorefname{Figure}
\def\subfigureautorefname{Figure}
\def\tableautorefname{Table}
\def\partautorefname{Part}
\def\appendixautorefname{Appendix}
\def\equationautorefname{Equation}
\def\Itemautorefname{Item}
\def\chapterautorefname{Chapter}
\def\sectionautorefname{Section}
\def\subsectionautorefname{Section}
\def\subsubsectionautorefname{Section}
\def\Hfootnoteautorefname{Footnote}
\def\AMSautorefname{Equation}
\def\theoremautorefname{Theorem}
\def\algorithmautorefname{Algorithm}

% autoref case is set here
%\def\figureautorefname{Fig.}
%\def\subfigureautorefname{Fig.}
%\def\tableautorefname{Table}
%\def\subtableautorefname{Table}
%\def\partautorefname{Part}
%\def\appendixautorefname{Appendix}
%\def\equationautorefname{Eq.}
%\def\Itemautorefname{Item}
%\def\chapterautorefname{Chapter}
%\def\sectionautorefname{Sec.}
%\def\subsectionautorefname{Sec.}
%\def\subsubsectionautorefname{Sec.}
%\def\Hfootnoteautorefname{Footnote}
%\def\AMSautorefname{Eq.}
%\def\theoremautorefname{Theorem}
%\def\algorithmautorefname{Algorithm}
%\renewcommand{\Autoref}{\autoref}

%\figurename         Figure
%\tablename 	     Table
%\partname           Part
%\appendixname       Appendix
%\equationname       Equation
%\Itemname           item
%\chaptername        chapter
%\sectionname        section
%\subsectionname     subsection
%\subsubsectionname  subsubsection
%\paragraphname      paragraph
%\Hfootnotename      footnote
%\AMSname            Equation
%\theoremname        Theorem
%\page 	             page


% Figures

% Dummy figure file
\def\dummyfigure{LaTeX/dummy}%

% Includegraphics wrapper macro to include either dummy or real figure
\newcommand{\incgfx}[2]{%
    \def\figfilename{\dummyfigure}%
    \def\testfile{\chapdir/Figures/#2}%
    \IfFileExists{\testfile.jpg}{\def\figfilename{\testfile}}{}%
    \IfFileExists{\testfile.png}{\def\figfilename{\testfile}}{}%
    \IfFileExists{\testfile.pdf}{\def\figfilename{\testfile}}{}%
    \IfFileExists{\testfile.jpeg}{\def\figfilename{\testfile}}{}%
    \IfFileExists{\testfile.tif}{\def\figfilename{\testfile}}{}%
    \IfFileExists{\testfile.tiff}{\def\figfilename{\testfile}}{}%
    \includegraphics[#1]{\figfilename}%
}%

% TODO: replace \incgfx with \imgRs
% TODO: fix \imgRs to convert only if source file exists and if destination doesn't
% TODO: fix \imgRs to re-sample more generically w/ width/height settings
% \imgRs{file}{extension}{width_in_pts}{height_in_pts}
\newcommand{\doConvert}[4]{\immediate\write18{convert #1.#2 -resample #3 -resize #4 #1_RS.#2}}
\newcommand{\imgRs}[4]{%
    \doConvert{#1}{#2}{#3}{#4}%
    \includegraphics[width=#3pt]{#1.rs}%
}

% Basic figure macro
% Arguments \fig[placement]{includegraphics opts}{filename}{short caption}{long caption}
% e.g. \fig{htbp}{width=10cm}{example}{Example Figure}{This is an example figure}
% If the file does not exist (with extension appropriate for your output document type),
% a dummy figure will be used instead
\newcommand{\fig}[5][htb]{
    \begin{figure}[#1]
        \begin{center}
            \incgfx{#2}{#3}
            \caption[#4]{#5}\label{fig:#3}
        \end{center}
    \end{figure}
}

%%%%%%%%%%%%%%%%%%%%%%%%%%%%%%%%%%%%
% Some other formatting stuff...

%% Chose your symbols for hierachical itemised lists
\renewcommand{\labelitemi}{$\bullet$}
\renewcommand{\labelitemii}{$\blacktriangleright$}
\renewcommand{\labelitemiii}{$\bigstar$}
\renewcommand{\labelitemiv}{$\blacklozenge$}

%% Use "List of References" instead of "Bibliography"?
% A List of References can contain only cited work; a Bibliography can contain
% entries in addition to cited work.
\renewcommand{\bibname}{List of References}


%%%%%%%%%%%%%%%%%%%%%%%%%
% Hypotheses
\declaretheorem[numberwithin=chapter,
    refname={Hypothesis,Hypotheses},
    Refname={Hypothesis,Hypotheses}]{hypothesis}

% Definitions
\declaretheorem[numberwithin=chapter,
    refname={Definition,Definitions},
    Refname={Definition,Definitions}]{definition}

% Proposition
\declaretheorem[numberwithin=chapter,
    refname={Proposition,Propositions},
    Refname={Proposition,Propositions}]{proposition}

% Theorems
\declaretheorem[numberwithin=chapter]{theorem}


%%%%%%%%%%%%%%%%%%%%%%%%%
% Framed Example Environment
% Choose a background colour (very light colours are best, but check how they print on 
% the printer that you'll be printing your final document on, or just stick to a light grey)
\definecolor{examplebackground}{cmyk}{0, 0.005, 0.06, 0.03}
% some basic layout
\newlength{\exmargin}   \setlength{\exmargin}{1em}
\newlength{\exlinewidth}\setlength{\exlinewidth}{2pt}
% the theorem style
\newtheoremstyle{thexample}% name
    {\topsep}%    Space above
    {\topsep}%    Space below
    {}%   Body font
    {}%           Header indent amount (empty = no indent, \parindent = para indent)
    {\bfseries}%  Thm head font
    {}% Punctuation after thm head
    {0pt}%   Space after thm head (\newline = linebreak)
    {\thmname{#1}\thmnumber{ #2}\thmnote{ --- #3}}% Thm head spec
    % \declaretheorem[parent=section, title=Example, style=example, preheadhook=\exprehead{}, postfoothook=\expostfoot{}]{example}
\declaretheorem[parent=chapter, title=Example, style=thexample]{thexample}
% re-def'd begin for syntax highlighting bug in TextMate (LaTeX package)
\def\thbegin{\begin}
% \def\exprehead{\thbegin{mdframed}[linewidth=1,margin=25]}%
% \def\expostfoot{\end{mdframed}}%
\newenvironment{example}[2][]% optional title, mandatory label (or something like it)
    {\thbegin{mdframed}
    [linewidth={\exlinewidth},leftmargin={\exmargin},rightmargin={\exmargin},backgroundcolor={examplebackground}]
    \thbegin{thexample}[#1]#2\hspace{0pt}\nopagebreak
    \setstretch{1.15}

    \nopagebreak}% guarantee something to break the line from (0pt space)
    {\end{thexample}\end{mdframed}}%
\makeatletter
%\newcommand{\listofexamples}{\@starttoc{loe}}
% HACK: fix up the line spacing in List of Examples/Theorems
\renewcommand{\l@thexample}[2]{\@dottedtocline{1}{1.5em}{2.3em}{#1}{#2}\vspace{-1.3\parskip}}
\makeatother
% horizontal bar completely across an example box (will need to adjust if margins change)
\newcommand{\exbar}{
\newlength{\barwidth}
\addtolength{\barwidth}{\textwidth}
\addtolength{\barwidth}{2\exmargin}
\addtolength{\barwidth}{\exlinewidth}
\newlength{\baroffset}
\addtolength{\baroffset}{\exmargin}
\addtolength{\baroffset}{0.5\exlinewidth}
\hspace{-\baroffset}\rule[.5\parskip]{\barwidth}{.5\exlinewidth}}



