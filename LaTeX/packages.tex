%!TEX root = ../Thesis.tex

% APA Citations
%\usepackage{apacite}

% Fonts and encodings...
% The standard Computer Modern fonts are in OT1 encoding (Type 3 fonts).
% Install the package cm-super to get Computer Modern fonts with T1 support.
% See http://tex.stackexchange.com/questions/1291/why-are-bitmap-fonts-used-automatically
%
% See http://tex.stackexchange.com/questions/1390/latin-modern-vs-cm-super where it says
%    "With cm-super it's recommended to use the fix-cm package to fix a lot of broken design
%     decisions in cm-super (and in addition this makes the final PDF a bit smaller)"
%
% As an alternative, see package ec - computer modern fonts in T1 and TS1 encodings
%\usepackage{cm-super}          % if your LaTeX distro does not include cm-super
%\usepackage{fix-cm}
\usepackage{textcase}
\usepackage{lmodern}
\usepackage{mathptmx} % Make the font Times New Roman equivalent
\usepackage[T1]{fontenc}
\usepackage[utf8]{inputenc}     % UTF-8 is a variable-width encoding (codepage) that can
                                % represent every character in the Unicode character set.
                                % It was designed for backward compatibility with ASCII

\usepackage{color}              % colours in text
\usepackage{enumitem}
\usepackage{fancybox,fancyhdr,setspace}     % layout packages
\usepackage{amsmath,amssymb,amsthm}         % Maths environments and symbols
\usepackage{thmtools}                       % Theorem tools 'list of' package (for AMS Theorem)

% Tables...
\usepackage{supertabular,array}             % multi-page table; array to properly space the table rows
\usepackage{booktabs}                       % "professional tables"
\usepackage{tabularx}                       % extra features for tables
%\usepackage{multirow}                      % multirow within tables
%\usepackage{colortbl}                      % coloured tables

% Caption formatting (must be before subfig and some other caption/figure related packages)
\usepackage[margin=10pt,font=small,labelfont=bf,indention=0.75cm,labelsep=endash]{caption}
\usepackage{float}                          % allow [H] placement for figures
\usepackage{ifthen}                         % much simpler IF THEN ELSE commands

\usepackage{subfig}                         % fancy sub-figures
\captionsetup[subfigure]{justification=centerlast,indention=0cm,margin=0pt}


% TODO: work out how to make \autoref{fig:subfig} format as #.##(x) instead of #.##x

\usepackage[bottom,stable]{footmisc}        % footnotes at bottom of page (rather than simply after the lowest text)
                                            % you can safely ignore the "LaTeX Warning: Command \@makecol  has changed."
                                            % according to the author of footmisc.

%\usepackage{varioref}                      % fancy page references, like 'on the next page'
%\usepackage{moreverb}                      % fancy verbatim features, like line numbering
\usepackage[normalem]{ulem}                 % Strikethrough \sout command
\usepackage{textcomp}                       % just for the trademark symbol \texttrademark
\usepackage{gensymb}                        % general symbols package (in text + math modes)
\usepackage[printonlyused]{acronym}         % acronyms package

% Referencing - natbib is nice
% Documentation at http://merkel.zoneo.net/Latex/natbib.php
%
% Bibliography styles supported by this template are apa-like (default) and ieee
% \ifthenelse{\equal{\tReferenceStyle}{apa-like}}
    % {\usepackage[square,comma,compress,numbers]{natbib}}   % IEEE-like style
    % {\usepackage[authoryear,round,sort]{natbib}}           % APA-like

\usepackage[algochapter,vlined,ruled]{algorithm2e}% algorithms / pseudocode
\SetAlgoSkip{bigskip}

\usepackage{listings}

\usepackage{dirtree}                        % just for the directory layout figure shown in the
                                            % chapter `Introduction'. You can probably remove this
                                            % if you don't want to show any file/dir-tree structures

% Graphics inclusion with hyperlinking in PDFs...
\usepackage[pdftex]{graphicx}
% TODO: consider backref or pagebackref options for hyperref (?)
\definecolor{urlcolour}{rgb}{0,0,0.6}

% Define link colours for easy editing - change them below for final document...
\usepackage[colorlinks=true, citecolor=red, linkcolor=blue, urlcolor=blue, bookmarks=true, bookmarksnumbered=true, pdftex, plainpages=false, pdfpagelabels]{hyperref}
% \usepackage{MyPackages/digsig} % TODO Add Signatures
% !! MUST define all link colours as black for final document
%\usepackage[colorlinks=true, citecolor=black, linkcolor=black, urlcolor=black, bookmarks=true, bookmarksnumbered=true, pdftex, plainpages=false, pdfpagelabels]{hyperref}

\hypersetup{pdfauthor={\pdfauthor}, pdftitle={\tdoctype}, pdfsubject={\ttitle},
    pdfkeywords={\tkeywords}}

\usepackage{rotating}                       % sideways tables and figures - this must come after graphicx

\usepackage[final]{pdfpages}                % Allows embedding PDFs as pages in the document.
% It appears that this package must be included AFTER the graphicx package.

% package: showkeys
% You can uncomment the following lines to make LaTeX print cross-referencing labels in your
% document, which may make cross-referencing easier if your editor can't autocomplete nicely.
%\usepackage[color]{showkeys}% show labels for proof-reading
%\renewcommand{\showkeyslabelformat}[1]{\fbox{\normalfont\tiny\ttfamily#1}}
%\definecolor{refkey}{gray}{0.9}
%\definecolor{labelkey}{gray}{0.9}

\usepackage[all]{hypcap}                    % hyperrefs to top of images (must be after hyperref and caption)
%\hypcapredef{algorithm}

\usepackage{mdframed}                       % TODO: consider [skipbelow=0pt]
%\usepackage{tikz}

%\usepackage{degrade}
%\DegSetup{res=300}

\usepackage{ragged2e}
\usepackage{titlecaps}
\usepackage[
    type={CC},
    modifier={by-nc},
    version={4.0},
]{doclicense}
% Can be deleted later
\usepackage{lipsum} 

